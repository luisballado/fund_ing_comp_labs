%%%%%%%%%%%%%%%%%%%%%%%%%%%%%%%%%%%%%%%%%
% fphw Assignment
% LaTeX Template
% Version 1.0 (27/04/2019)
%
% This template originates from:
% https://www.LaTeXTemplates.com
%
% Authors:
% Class by Felipe Portales-Oliva (f.portales.oliva@gmail.com) with template 
% content and modifications by Vel (vel@LaTeXTemplates.com)
%
% Template (this file) License:
% CC BY-NC-SA 3.0 (http://creativecommons.org/licenses/by-nc-sa/3.0/)
%
%%%%%%%%%%%%%%%%%%%%%%%%%%%%%%%%%%%%%%%%%

%----------------------------------------------------------------------------------------
%	PACKAGES AND OTHER DOCUMENT CONFIGURATIONS
%----------------------------------------------------------------------------------------

\documentclass[
	12pt, % Default font size, values between 10pt-12pt are allowed
	%letterpaper, % Uncomment for US letter paper size
	%spanish, % Uncomment for Spanish
]{fphw}

% Template-specific packages
\usepackage[utf8]{inputenc} % Required for inputting international characters
\usepackage[T1]{fontenc} % Output font encoding for international characters
\usepackage{mathpazo} % Use the Palatino font
\usepackage[dvipsnames]{xcolor}
\usepackage{graphicx} % Required for including images
\usepackage{amsmath}
\usepackage{booktabs} % Required for better horizontal rules in tables
\usepackage{listings} % Required for insertion of code
\usepackage{enumerate} % To modify the enumerate environment
\usepackage{ragged2e}
\usepackage{cancel}
\usepackage{MnSymbol,bbding,pifont}
\usepackage{lscape}
\usepackage{array}
\usepackage{float,graphicx}
\usepackage{hyperref}
\usepackage[spanish]{babel}
\newcolumntype{M}{>{$}c<{$}}
%----------------------------------------------------------------------------------------
%	ASSIGNMENT INFORMATION
%----------------------------------------------------------------------------------------

\title{Ensayo \#2} % Assignment title

\author{Luis Alberto Ballado Aradias} % Student name

\date{\today} % Due date

\institute{Centro de Investigación y de Estudios Avanzados del IPN \\ Unidad Tamaulipas} % Institute or school name

\class{CONTROL AUTOMÁTICO (Sep - Dec 2022)} % Course or class name

\professor{Dr. José Gabriel Ramírez Torres} % Professor or teacher in charge of the assignment

%----------------------------------------------------------------------------------------

\begin{document}

\maketitle % Output the assignment title, created automatically using the information in the custom commands above

%----------------------------------------------------------------------------------------
%	ASSIGNMENT CONTENT
%----------------------------------------------------------------------------------------
\section*{{\color{Apricot}Routh-Hurwitz Criterion, An Introduction} \url{https://www.youtube.com/watch?v=WBCZBOB3LCA}}

Para que un sistema sea considerado estable todas las raíces del polinomio característico necesitan situarse del lado izquierdo del plano complejo. \\
Considerando que la ecuación característica está en el denominador de la función de transferencia, las raíces son las mismas que los polos de la función de transferencia y esas raíces deben aparecer a la izquierda o ser todas componentes reales negativos para tener un sistema estable. \\
Un ejemplo podría ser, considerar una función con un solo polo $H(s) = \frac{1}{s+a}$ a puede tomar valores positivos o negativos; en nuestro ejemplo la raíz es negativa, podríamos aplicar la inversa de Laplace para obtener la representación en el dominio del tiempo ${L}^{-1} (H(s)) = e^{-at}$ al tener una raíz negativa, existirá en la parte izquierda del plano y si se grafica en el dominio del tiempo, se puede apreciar que la señal tiende a cero a medida que el tiempo tiende a infinito y es considerado estable ya que ninguna otra función de transferencia modificara su comportamiento. Si, de lo contrario; la función fuera $H(s)=\frac{1}{s-a}$  \textbf{s} seria positiva, la respuesta seria positiva y el sistema seria inestable.
Considerando una función de transferencia más completa, es decir con más raíces, no importa que por partes sean estables, al detectar una parte que no tenga raíces del lado izquierdo del plano todo el sistema se considera inestable.
Podemos determinar la estabilidad de un sistema para cualquier grado del polinomio sin resolverlo directamente y una forma de realizarlo es usando el criterio de Routh-Hurwitz. Este método se basa en que todas las raíces del polinomio se encuentran del lado izquierdo del plano si y solo si la combinación de sus coeficientes tiene el mismo signo. Lo notable del criterio del Routh-Hurwitz es que no se tiene que resolver la factorización para encontrar las raíces del polinomio característico, y nos permite conocer la estabilidad solo en ver los signos de los coeficientes del polinomio.
Lo primero que debemos notar es el signo de los coeficientes, si todos los signos no son iguales podemos decir q con certeza que el sistema es inestable. 
Si contamos con una ecuación característica con al menos un coeficiente no negativo, podemos instantáneamente decir que el sistema es inestable sin tener que aplicar el criterio de Routh-Hurwitz.
$G(s)=\frac{1}{s+1}\ * \frac{1}{s+3} * \frac{1}{s-2}$ \\

\section{Routh Array}

Es una tabla que se pobla con los coeficientes del polinomio siguiendo unas reglas.

\begin{enumerate}
\item{Hacer la estructura de la tabla, al completar la tabla de izquierda a derecha, el número de las filas depende del orden del polinomio.  Para comenzar necesitamos escribir el polinomio en potencias de s de forma descendente. Las filas se acomodan de manera que la de mayor orden quedara arriba hasta decrecer el orden. El número de columnas depende del orden del polinomio.}
\item{Escribir el primer coeficiente, y así sucesivamente siguiendo el orden del polinomio.}
\end{enumerate}

El orden será de forma zigzag. Al final tendremos al inicio los coeficientes impares y por debajo los impares.\\
Nota: Si no se cuenta con un orden en el polinomio, es decir $s^4 + 2s^3 + 5s + 3$ deberemos completarlo con un 0 en el orden que haga falta, en este caso $s^2$ es cero, no es que no exista, solo que es cero.\\

\textbf{Hacer un ejemplo aqui}

Una vez que tengamos completa la tabla, se puede visualizar las raíces y los cambios de signo con los que cuenta el sistema, recordemos que basta con uno para que el sistema sea considerado inestable. \\

\newpage
\section*{{\color{Apricot}Routh-Hurwitz Criterion, Special Cases} \url{https://www.youtube.com/watch?v=oMmUPvn6lP8}}


\end{document}
