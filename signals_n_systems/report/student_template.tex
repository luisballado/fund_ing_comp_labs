%%%%%%%%%%%%%%%%%%%%%%%%%%%%%%%%%%%%%%%%%
% Cleese Assignment (For Students)
% LaTeX Template
% Version 2.0 (27/5/2018)
%
% This template originates from:
% http://www.LaTeXTemplates.com
%
% Author:
% Vel (vel@LaTeXTemplates.com)
%
% License:
% CC BY-NC-SA 3.0 (http://creativecommons.org/licenses/by-nc-sa/3.0/)
% 
%%%%%%%%%%%%%%%%%%%%%%%%%%%%%%%%%%%%%%%%%

%----------------------------------------------------------------------------------------
%	PACKAGES AND OTHER DOCUMENT CONFIGURATIONS
%----------------------------------------------------------------------------------------

\documentclass[11pt]{article}

\input{structure.tex} % Include the file specifying the document structure and custom commands

%----------------------------------------------------------------------------------------
%	ASSIGNMENT INFORMATION
%----------------------------------------------------------------------------------------

% Required
\newcommand{\assignmentQuestionName}{Pregunta} % The word to be used as a prefix to question numbers; example alternatives: Problem, Exercise
\newcommand{\assignmentClass}{Señales y Sistemas} % Course/class
\newcommand{\assignmentTitle}{Examen} % Assignment title or name
\newcommand{\assignmentAuthorName}{Luis Alberto Ballado Aradias} % Student name

% Optional (comment lines to remove)
\newcommand{\assignmentClassInstructor}{Dr. José Juan García Hernández} % Intructor name/time/description
\newcommand{\assignmentDueDate}{CINVESTAV - UNIDAD TAMAULIPAS} % Due date

%----------------------------------------------------------------------------------------

\begin{document}

%----------------------------------------------------------------------------------------
%	TITLE PAGE
%----------------------------------------------------------------------------------------

\maketitle % Print the title page

\thispagestyle{empty} % Suppress headers and footers on the title page

\newpage

%----------------------------------------------------------------------------------------
%	QUESTION 1
%----------------------------------------------------------------------------------------

\begin{question}

\questiontext{Dibuja las siguientes señales}

\begin{enumerate}
\item{$x[n] = u[n+3] + 0,5u[n-1]$}
  \begin{center}
    \includegraphics[width=0.5\columnwidth]{graph0.png} % Example image
  \end{center}
\item{$x[n] = -1^{n} u[-n-2]$}
  \begin{center}
    \includegraphics[width=0.5\columnwidth]{graph1.png} % Example image
  \end{center}
\item{$x[n] = \sum_{i=0}^{\infty} 4\delta[n-3k-1] $}
  \begin{center}
    \includegraphics[width=0.5\columnwidth]{graph2.png} % Example image
  \end{center}
\end{enumerate}

\end{question}

%----------------------------------------------------------------------------------------
%	QUESTION 2
%----------------------------------------------------------------------------------------

\newpage
\begin{question}

\questiontext{Describa todas las características que sean evidentes de los siguientes sistemas}

%--------------------------------------------
\begin{enumerate}
\item{$y[n] = 3x[n-1] + 2x[n-2] + 0,75x[n+4] - 3y[n-1]$}

  \answer{
  \begin{itemize}
  \item Es un Sistema Lineal
  \item El valor de la salida depende de valores futuros de la entrada, el sistema \textbf{tiene memoria} 
  \item Debido a que la salida depende de valores futuros de la entrada el sistema \textbf{no es causal}
  \item \textbf{Sistema Inestable} por la retroalimentación
  \item Variante en el tiempo
  \end{itemize}
  }
\item{$y[n] = x[n] cos\left[ \frac{n}{2\pi}\right]$}
  \answer{
  \begin{itemize}
  \item Sistema No lineal, tiene una función periódica
  \item Invariante en el tiempo
  \item Los valores de salida n dependen solo de valores de entrada en el momento n, \textbf{sistema sin memoria}
  \item La salida no depende de valores futuros, el sistema \textbf{es causal}
  
  \end{itemize}
  }
  
\item{$y[n] = 2n^{2}x[n] + n \times x[n+1]$}

  \answer{
  \begin{itemize}
  \item \textbf{No} es un Sistema Lineal por el termino cuadratico
  \item El valor de la salida depende de valores futuros de la entrada, el sistema \textbf{tiene memoria} 
  \item Debido a que la salida depende de valores futuros de la entrada el sistema \textbf{no es causal}
  \item \textbf{Sistema Inestable}
  \item Variante en el tiempo
  \end{itemize}
  }
  
\end{enumerate}

%--------------------------------------------

\end{question}
\newpage
%----------------------------------------------------------------------------------------
%	QUESTION 3
%----------------------------------------------------------------------------------------

\begin{question}

\questiontext{Calcular la transformada Z de las 3 señales y los 3 sistemas previamente descritos.}

\begin{equation}\label{eq:uno}
  x[n]=u[n+3]+0,5u[n-1]
\end{equation}

\answer{
  \[X[z] = x[n]\cdot z^{-n}\]
  \[X[z] = \sum_{k=-3}^{-\infty} 1\cdot z^{-k} + 0,5 \sum_{k=1}^{\infty} 1 \cdot z^{-k}\]
  \[X[z] = \sum_{n=-3}^{-\infty} \left(\frac{1}{z}\right)^{n}+0,5 \sum_{n=1}^{\infty} \left(\frac{1}{z}^{n}\right)\]
  A partir de la serie geométrica:
  \[ \sum_{n=0}^{N} r^{n} = \frac{1-r^{N+1}}{1-r} \implies  \frac{\left(\frac{1}{z}\right)^{-3}-0}{1-\frac{1}{z}}+\frac{0,5\left(\frac{1}{z}\right)^{1}-0}{1-\frac{1}{z}}\]
  \[X[z]=\frac{z^3}{1-z^{-1}}+\frac{0.5\cdot z^{-1}}{1-z^{-1}}\]
  \[= \frac{z^3+0,5\cdot z^{-1}}{1-z^{-1}}\]
  expresando en positivos
  \[= \frac{z^3+0,5\cdot z^{-1}}{1-z^{-1}}\cdot\frac{z}{z}=\frac{z^4+0,5}{z-1}\]
}

\begin{equation}\label{eq:dos}
  x[n]=-1^n u[-n-2]
\end{equation}

\answer{
  \[X[z] = -1^n u[-n-2]\cdot z^{-n}\]
  \[X[z] = \sum_{n=-2}^{-\infty} -1^n\cdot z^{-n} = \sum_{n=-2}^{-\infty}-\left(\frac{1}{z}\right)^n\]
  A partir de la serie geométrica:
  \[X[z]=\frac{\left(-\frac{1}{z}\right)^{-2}-0}{1-\left(-\frac{1}{z}\right)}=\frac{\frac{1}{z^{-2}}}{1+z^{-1}}=\frac{z^2}{1+z^{-1}}\cdot \frac{z}{z}\]
  expresando en positivos
  \[X[z]= \frac{z^3}{z+1}\]
}

\newpage
\begin{equation}\label{eq:tres}
  x[n]=\sum_{k=0}^{\infty} 4\delta[n-3k-1]
\end{equation}

\answer{
  \[X[z]=4 \sum_{k=0}^{\infty} z^{-3k-1} = 4\sum_{k=0}^{\infty} \left(\frac{1}{z}\right)^{-3k-1}\]
  A partir de la serie geométrica:
  \[X[z]=4\left(\frac{\left(\frac{1}{z}\right)^{-(3\cdot0)-1}}{1-\frac{1}{z}}\right)=\frac{4z}{1-z^{-1}\cdot\frac{z}{z}}=\frac{4z^2}{z-1}\]
}

\begin{equation}\label{eq:cuatro}
  y[n]=3x[n-1]+2x[n-2]+0.75x[n+4]-3y[n-1]
\end{equation}

\answer{
  \[y[n]+3y[n-1]=3x[n-1]+2x[n-2]+0.75x[n+4]\]
  \[Y[z]+3Y[z]\cdot z^{-1} = 3X[z]\cdot z^{-1}+2X[z]\cdot z^{-2}+0.75X[z]\cdot z^4\]
  \[Y[z]\left(1+\frac{3}{z}\right)=X[z]\left(\frac{3}{z}+\frac{2}{z^{-2}}+0.75z^4\right)\]
  \[\frac{Y[z]}{X[z]}=\frac{3z^{-1}+2z^{-2}+0.75z^4}{1+3z^{-1}}\cdot \frac{z}{z}=\frac{3+2z+0.75z^5}{z+3}\]
}

\begin{equation}\label{eq:cinco}
  y[n]=x[n] cos\left[\frac{n}{2\pi}\right]
\end{equation}

\answer{
  \[Y[z]=X[z] z^0 \cdot \frac{z^2-z\cdot cos\left[\frac{n}{2\pi}\right]}{z^2-2z(cos\left[\frac{n}{2\pi}\right])+1}\]
  \[\frac{Y[z]}{X[z]}=\frac{z^2-z\cdot cos\left[\frac{n}{2\pi}\right]}{z^2-2z(cos\left[\frac{n}{2\pi}\right])+1}\]
}

\begin{equation}\label{eq:seis}
  y[n]=2n^2\cdot x[n] + n\cdot x[n+1]
\end{equation}

\answer{
  \[Y[z]=2n^2 X[z] z^{-0} + nX[z]z^1 = X[z](2n^2+zn)\]
  \[\frac{Y[z]}{X[z]}=2n^2+zn\]
}

\end{question}

%----------------------------------------------------------------------------------------

%\assignmentSection{Bonus Questions}

%----------------------------------------------------------------------------------------
%	QUESTION 4
%----------------------------------------------------------------------------------------
\newpage
\begin{question}

\questiontext{Describa que es una eigenfunción en términos de señales y sistemas.}

\answer{Lorem ipsum dolor sit amet, consectetur adipiscing elit. Praesent porttitor arcu luctus, imperdiet urna iaculis, mattis eros. Pellentesque iaculis odio vel nisl ullamcorper, nec faucibus ipsum molestie. Sed dictum nisl non aliquet porttitor. Etiam vulputate arcu dignissim, finibus sem et, viverra nisl. Aenean luctus congue massa, ut laoreet metus ornare in. Nunc fermentum nisi imperdiet lectus tincidunt vestibulum at ac elit. Nulla mattis nisl eu malesuada suscipit.}

\end{question}

\begin{question}

\questiontext{Describa las características particulares de la transformada de Laplace.}
\answer{Lorem ipsum dolor sit amet, consectetur adipiscing elit. Praesent porttitor arcu luctus, imperdiet urna iaculis, mattis eros. Pellentesque iaculis odio vel nisl ullamcorper, nec faucibus ipsum molestie. Sed dictum nisl non aliquet porttitor. Etiam vulputate arcu dignissim, finibus sem et, viverra nisl. Aenean luctus congue massa, ut laoreet metus ornare in. Nunc fermentum nisi imperdiet lectus tincidunt vestibulum at ac elit. Nulla mattis nisl eu malesuada suscipit.}
\end{question}
\newpage
\begin{question}

  \questiontext{Realice un programa, en cualquier lenguaje que prefiera, que ejecute las siguientes tareas.}
  
  \begin{itemize}
  \item{Recibe como entrada en texto plano la descripción de una señal y de un sistema, discretos ambos.}
  \item{Dibuja la señal y la respuesta al impulso del sistema.}
  \item{Ejecuta la convolución entre ambas entradas y dibujar la señal resultante.}
  \end{itemize}

  \lstinputlisting[
	caption=Luftballons Perl Script, % Caption above the listing
	label=lst:luftballons, % Label for referencing this listing
	language=Perl, % Use Perl functions/syntax highlighting
	frame=single, % Frame around the code listing
	showstringspaces=false, % Don't put marks in string spaces
	numbers=left, % Line numbers on left
	numberstyle=\tiny, % Line numbers styling
	]{luftballons.pl}

  \begin{subquestion}{How many luftballons will be output by the Listing \ref{lst:luftballons} above?} % Subquestion within question

\answer{99 luftballons.}

\end{subquestion}

  
%--------------------------------------------



%--------------------------------------------

\begin{subquestion}{Identify the regular expression in Listing \ref{lst:luftballons} and explain how it relates to the anti-war sentiments found in the rest of the script.} % Subquestion within question

\answer{Lorem ipsum dolor sit amet, consectetur adipiscing elit. Praesent porttitor arcu luctus, imperdiet urna iaculis, mattis eros. Pellentesque iaculis odio vel nisl ullamcorper, nec faucibus ipsum molestie. Sed dictum nisl non aliquet porttitor. Etiam vulputate arcu dignissim, finibus sem et, viverra nisl. Aenean luctus congue massa, ut laoreet metus ornare in. Nunc fermentum nisi imperdiet lectus tincidunt vestibulum at ac elit. Nulla mattis nisl eu malesuada suscipit.}

\end{subquestion}
  
\end{question}

\end{document}
